\section{State of Art}\label{sec:relatedwork}
Highly automated vehicles are typically \emph{learning-enabled cyber physical systems} operating in an uncertain environment, where the learning about the \emph{dynamic} environment is enabled through \emph{inaccurate} sensors. This renders an exact inference of the state of the environment infeasible, necessitating representations of the \emph{uncertainty} in inferring about a dynamic environment through inaccurate sensors. Such representations of uncertainty have been investigated a.o. within the paradigm of \emph{probabilistic robotics}~\cite{probrob}, particularly as applied to vehicle localization in urban environments~\cite{Thrun2007,Thrun2009,Thrun2010}. In these and related works such as~\cite{Perrolaz2014, moras2014}, the environment uncertainty is represented as \emph{probabilistic
beliefs}. 


As an example, consider a popular perception technique like simultaneous localization and
mapping [5] (SLAM), which is used for determining the current position of an autonomous
vehicle. The estimated position of the vehicle and the coordinates of other entities in the map
are often assumed to have Gaussian noise. Aside from localization and mapping, another critical
perception challenge for autonomous vehicles is obstacle detection and tracking [8,27].
Camera and laser range finders are used to locally detect and avoid obstacles during navigation
for a previously constructed map. This is particularly useful in the presence of dynamic
objects whose locations are not fixed in the environment map. The uncertainty in the parametric
models representing the obstacles is usually also modelled using Gaussian random
variables




